% INTRODUÇÃO-------------------------------------------------------------------

\chapter{INTRODUÇÃO}
\label{chap:introducao}

% Historia da Pesca
A pesca é uma das atividades mais antigas da humanidade, datando de dezenas de milhares de anos. Sua prática, que consiste na captura de animais aquáticos, tem como objetivo principal a alimentação. Segundo \citeonline{afonso2007breves}, vestígios sugerem que ela já era praticada no período Paleolítico, há cerca de 50 mil anos, e evidências de tal atividade foram encontradas em pinturas rupestres no continente Africano e Europeu, com mais de 25 mil anos.

% Evolução da pesca no tempo
Ao longo da evolução da pesca, diversos artefatos foram desenvolvidos para aumentar a sua facilidade e eficiência, como documentado por \citeonline{afonso2007breves}, itens como redes, linhas, anzóis, arpões, flutuadores e pesos foram criados para otimizar a atividade.

% Chegada da tecnologia na Pesca e descrever beneficios
Com o avanço tecnológico, a pesca se beneficiou da criação de produtos mais complexos e eficientes, como barcos mais bem equipados, sonares para localizar peixes com facilidade, varas de pesca mais complexas, entre outros artefatos e práticas que auxiliam a atividade, conforme detalha \citeonline{oliveira2020v}.

Nos dias atuais, a presença da internet proporcionou que novos recursos e convenções fossem criados para aumentar ainda mais a eficiencia da pesca. Como descrito por \citeonline{nassiff2025frota} a introdução de recursos de Internet das Coisas (IoT) para monitoramento e tambem o uso de inteligência artificial (IA) para mapeamento preciso dos cordumes de peixes são exemplos desses avanços.

Como aborado por \citeonline{skov2021expert}, outros recursos modernos, como aplicativos móveis e plataformas web, também estão começando a contribuir para aumentar o compartilhamento de informações entre a comunidade de pescadores, com o intuito de melhorar os costumes pesqueiros e aumentar e eficiencia.

% Motivo do desenvolvimento do trabalho
Com o objetivo de melhorar a eficiência dos pescadores, este trabalho propõe o desenvolvimento de um aplicativo móvel com o intuito de ajudar aqueles que têm dificuldade em encontrar bons locais de pesca, seja em áreas remotas ou em locais controlados, como pesqueiros. A solução visa disponibilizar um mapa como recurso visual, onde será possível visualizar os pontos de pesca registrados por outros usuários. Esses pontos de pesca registrados conterão detalhes sobre a localização e sobre as espécies de peixes capturados.

Serão abordados os objetivos gerais e específicos do projeto, bem como a justificativa para sua relevância e escolha do tema. Em seguida, a metodologia será detalhada, apresentando a abordagem e a natureza do trabalho. O referencial teórico discorrerá sobre as tecnologias utilizadas, enquanto a seção de desenvolvimento abordará a documentação do projeto, incluindo diagramas de estrutura, requisitos e a prototipagem. Por fim, serão apresentados e discutidos os resultados obtidos.
