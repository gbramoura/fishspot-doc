% METODOLOGIA------------------------------------------------------------

\chapter{METODOLOGIA}
\label{chap:metodologia}

O presente trabalho é considerado uma pesquisa de natureza aplicada, visto que procura, através da análise de alguns recursos, desenvolver o conhecimento para aplicar uma solução para o problema apontado neste documento.

Considerando seus objetivos, o presente documento se caracteriza como uma pesquisa descritiva e exploratória, utilizando-se fontes acadêmicas para ser fundamentada e também buscando sondar o problema descrito neste documento por meio de formulários e entrevistas com a comunidade.

Quanto à forma de abordagem, este projeto se enquadra em pesquisa qualitativa, pois como abordado por \citeonline{godoy1995introduccao}, o que é valorizado na pesquisa qualitativa é o contato direto e prolongado do pesquisador com o ambiente e situação que está sendo estudada, e também descreve que um fenômeno pode ser bem mais observado e compreendido no contexto em que ocorre. Desse modo, a pesquisa poderá identificar não só a dificuldade que os pescadores possuem ao encontrar um local de pesca, mas também as formas e meios utilizados para localizar um bom local longe de perigos.

O tipo de procedimento técnico utilizado para este trabalho será levantamento, pois de acordo com o \citeonline{de2023metodologia} a pesquisa de levantamento é uma estratégia utilizada para levantar dados e informações sobre características ou opiniões de um grupo. Contudo, devido à necessidade da leitura de documentos para assegurar que as informações levantadas estão de fato pautadas em verdade, esta pesquisa conta com um âmbito documental.

% \section{COLETA E TRATAMENTO DE DADOS}
% \label{sec:titSecColDad}

% A principal forma de coleta de dados será por meio de formulários e questionários, que permitirão reunir informações de forma estruturada e padronizada. Os formulários poderão ser físicos ou digitais, facilitando a coleta em larga escala e garantindo uniformidade nas respostas, enquanto os questionários serão mais detalhados, abordando questões específicas relacionadas ao tema da pesquisa.

% Os formulários serão utilizados em locais com grande fluxo de pessoas, como pesqueiros, parques os quais possuem tanques de pesca e locais de acampamento com lagos, desta forma facilitando a coleta dos dados. Os questionários serão utilizados em locais onde perguntas mais detalhadas devem ser feitas, como exemplo rios e lagos localizados longe do perímetro urbano, córregos que possuem tratamento e a vida nativa é saudável, praias, pescas em alto-mar e dentre outros locais de pesca.

% A análise dos resultados obtidos nos questionários será executada de forma qualitativa, e poderá ser identificado quais são as dificuldades encontradas durante o processo de localização do melhor local de pesca.
