\chapter{JUSTIFICATIVA}
\label{chap:justificativa}

Atualmente, é comum a utilização de aplicativos para o preparo de uma rotina de pesca. A facilidade de encontrar locais de pescaria, bem como a visualização de detalhes sobre os peixes presentes e as iscas utilizadas, trazem segurança e tranquilidade para o cotidiano de um pescador.

Embora existam várias aplicações com esse propósito, é raro possuírem experiência de usuário (UX, \textit{User Experience}) de acordo com as necessidades do usuário comum. Frequentemente disponíveis sem tradução completa para a língua portuguesa, bem como faltando simplicidade em sua interface, o uso das ferramentas que buscam ser compreensíveis se torna algo complexo, de difícil aprendizado, impedindo o desfruto de momentos de lazer proporcionados pela pesca ao pescador.

Dessa forma, este aplicativo busca apoiar a colaboração da comunidade de pescadores para aumentar a eficiência durante o processo, permitindo que os usuários registrem os locais visitados, as dificuldades enfrentadas para chegar até esses pontos, os riscos encontrados durante a pesca e as espécies de peixes capturadas no período de lazer de forma fácil e compreensível, eliminando as dificuldades decorrentes de uma interface confusa e proporcionando aos usuários uma experiência tranquila e acessível.

O objetivo final desse projeto é proporcionar aos membros mais agilidade e simplicidade na busca por locais ideais, reduzindo o tempo necessário na pesquisa e, assim, trazendo maior eficácia na captura de bons exemplares de peixes e mais segurança durante todo o processo.

% ---

