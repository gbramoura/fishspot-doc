\begin{table}[h!]
    \centering
    \caption[Requisitos para Visualização Pontos de Pesca]{Requisitos para Visualização de Pontos de Pesca
    \label{tab:tb-rf-visualizar-pontos-pesca}}
    \setlength{\extrarowheight}{2pt}
    \begin{tabular}{|l!{\vrule width 1pt}p{0.55\textwidth}|c|c|}
        \hline
        \textbf{RF-2} & \multicolumn{3}{|l|}{Visualizar Pontos de Pesca} \\
        \hline
        \multicolumn{4}{|p{0.9\textwidth}|}{\textbf{Descrição detalhada:} } \\
        \hline
        \multicolumn{4}{|c|}{\textbf{REQUISITOS NÃO FUNCIONAIS ASSOCIADOS}} \\
        \hline
        \textbf{Nome} & \textbf{Restrição} & \textbf{Categoria} & \textbf{Desejável} \\
        \hline
        RNF2.1 & Durante visualização dos pontos de pesca, dados sensíveis dos outros usuários não devem ser expostos & Segurança & Obrigatório \\
        \hline
        RNF2.2 & Os pontos de pesca devem ser exibidos em um mapa, semelhante às aplicações Waze, Google Maps e etc. & Usabilidade & Obrigatório \\
        \hline
        RNF2.3 & Deve ser possivel visualizar os pontos de pesca de forma singular, exibindo detalhes do registro feito & Usabilidade & Obrigatório \\
        \hline
        RNF2.4 & Deve ser possivel visualizar as fotos registradas nos detalhes do ponto de pesca  & Usabilidade & Desejável \\
        \hline
        RNF2.5 & As informações devem ser exibidas de forma fluidas e o mapa não deve demorar para renderizar os pontos de pesca & Desempenho & Obrigatório \\
        \hline
    \end{tabular}
    \fonte{Fonte: Produzido pelo Autor}
\end{table}